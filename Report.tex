% Options for packages loaded elsewhere
\PassOptionsToPackage{unicode}{hyperref}
\PassOptionsToPackage{hyphens}{url}
%
\documentclass[
  11pt,
]{article}
\usepackage{amsmath,amssymb}
\usepackage{lmodern}
\usepackage{iftex}
\ifPDFTeX
  \usepackage[T1]{fontenc}
  \usepackage[utf8]{inputenc}
  \usepackage{textcomp} % provide euro and other symbols
\else % if luatex or xetex
  \usepackage{unicode-math}
  \defaultfontfeatures{Scale=MatchLowercase}
  \defaultfontfeatures[\rmfamily]{Ligatures=TeX,Scale=1}
\fi
% Use upquote if available, for straight quotes in verbatim environments
\IfFileExists{upquote.sty}{\usepackage{upquote}}{}
\IfFileExists{microtype.sty}{% use microtype if available
  \usepackage[]{microtype}
  \UseMicrotypeSet[protrusion]{basicmath} % disable protrusion for tt fonts
}{}
\usepackage{xcolor}
\IfFileExists{xurl.sty}{\usepackage{xurl}}{} % add URL line breaks if available
\IfFileExists{bookmark.sty}{\usepackage{bookmark}}{\usepackage{hyperref}}
\hypersetup{
  pdftitle={Estimating Support for Military Intervention by State Using Multilevel Regression with Post-Stratification},
  pdfauthor={Parker Kaukl},
  hidelinks,
  pdfcreator={LaTeX via pandoc}}
\urlstyle{same} % disable monospaced font for URLs
\usepackage[margin=1in]{geometry}
\usepackage{longtable,booktabs,array}
\usepackage{calc} % for calculating minipage widths
% Correct order of tables after \paragraph or \subparagraph
\usepackage{etoolbox}
\makeatletter
\patchcmd\longtable{\par}{\if@noskipsec\mbox{}\fi\par}{}{}
\makeatother
% Allow footnotes in longtable head/foot
\IfFileExists{footnotehyper.sty}{\usepackage{footnotehyper}}{\usepackage{footnote}}
\makesavenoteenv{longtable}
\usepackage{graphicx}
\makeatletter
\def\maxwidth{\ifdim\Gin@nat@width>\linewidth\linewidth\else\Gin@nat@width\fi}
\def\maxheight{\ifdim\Gin@nat@height>\textheight\textheight\else\Gin@nat@height\fi}
\makeatother
% Scale images if necessary, so that they will not overflow the page
% margins by default, and it is still possible to overwrite the defaults
% using explicit options in \includegraphics[width, height, ...]{}
\setkeys{Gin}{width=\maxwidth,height=\maxheight,keepaspectratio}
% Set default figure placement to htbp
\makeatletter
\def\fps@figure{htbp}
\makeatother
\setlength{\emergencystretch}{3em} % prevent overfull lines
\providecommand{\tightlist}{%
  \setlength{\itemsep}{0pt}\setlength{\parskip}{0pt}}
\setcounter{secnumdepth}{-\maxdimen} % remove section numbering
\usepackage{booktabs}
\usepackage{longtable}
\usepackage{array}
\usepackage{multirow}
\usepackage{wrapfig}
\usepackage{float}
\usepackage{colortbl}
\usepackage{pdflscape}
\usepackage{tabu}
\usepackage{threeparttable}
\usepackage{threeparttablex}
\usepackage[normalem]{ulem}
\usepackage{makecell}
\usepackage{xcolor}
\ifLuaTeX
  \usepackage{selnolig}  % disable illegal ligatures
\fi

\title{Estimating Support for Military Intervention by State Using
Multilevel Regression with Post-Stratification}
\author{Parker Kaukl}
\date{5/31/2022}

\begin{document}
\maketitle

\hypertarget{abstract}{%
\subsection{Abstract}\label{abstract}}

One of the most complex topics in American politics is military
intervention. Public opinion on military intervention is historically
divided into pro-war and anti-war positions. Understanding which states
may be more or less likely to be pro-war is important for predicting
which areas may have more protests, and inform policy positions for
elected officials. This paper will create a model to predict opinions on
military intervention using a Frequentist and a Bayesian approach. Then,
these models will be used in order to create a multilevel regression
with post-stratification to estimate the public opinion within each
state in the United States of America. In both the Bayesian and the
Frequentist approaches, our multilevel regression with
post-stratification estimates that the public opinion within each state
to believe that intervention is effective is between 60\% and 70\%.
These results tend to be similar to the public opinion polls on military
interventions throughout the late 2010s. \newpage

\hypertarget{introduction}{%
\subsubsection{Introduction}\label{introduction}}

The decision to go to war is a complex decision that has severe
consequences. A report from Brown University estimated that around
900,000 deaths occurred in the War on Terror. These fatalities include
soldiers, civilians, journalists, and aid workers. In addition, the
U.S.'s War on Terror has incurred a cost of eight trillion dollars since
2001. Because of these high societal costs, it is not surprising that
conflict tends to be a polarizing topic and a topic that has attracted
theories on when conflict may occur.

One of the first theories of conflict is the ``Just War'' theory, by
St.~Augustine. This posited that conflict should occur if it is for a
worthy cause, if conflict is likely to achieve that cause and if the
conflict is sponsored by a legitimate government. ``Just War'' theory
also states that conflict should only be used as a last resort, that
conflict should not be disproportionate to the cause, and that
non-combatants should be respected. Although this theory is valuable,
the discourse on military conflict has been furthered by philosophers,
politicians, and theologians over the past centuries.

However, perhaps the most relevant and important influence on our
perception of conflict has been the United States' recent history of
war. A lot of theories and thoughts on war were developed in the
aftermath of the Vietnam War. As noted in Richard Haas's book,
Intervention, the Vietnam War was an event that informs both our
policymakers and citizens on what war is like. Perhaps more importantly,
the Vietnam War and the protests related to U.S. involvement introduced
the importance of U.S. public opinion to whether we should engage in
military conflicts.

Caspar Weinberger, a Secretary of Defense under Ronald Reagan,
introduced the Weinberger Doctrine, which stated that force should only
be used if there is public or congressional support for military
conflict. This was furthered by Warren Christopher, the Secretary of
Defense under Bill Clinton, who said that having ``a high likelihood of
support'' of popular and congressional support was a prerequisite for
military intervention. Other politicians, such as George H.W. Bush, have
not mentioned public opinion as a prerequisite, but have mentioned that
it is desirable to have support for any military intervention.

The effects of public opinion during wartime are numerous. As noted by
George Gallup, public opinion can often coincide with political changes.
As stated in \emph{How Important is Public Opinion in Time of War?},
``In addition to conscription, the people of this country have been
ahead of their political leaders on virtually all important war-time
issues'' (Gallup, 441). In addition, conflict is often likely to not
occur, or forces are likely to be withdrawn in the case of public
disapproval. This can be seen in a lack of approval for U.S.
intervention in Somalia, and later the U.S.'s lack of military
intervention in Rwanda during the Rwandan Genocide. In addition, the
mismanagement of military intervention threatens re-election campaigns
for politicians. Because of this threat, ``Politicians--Whose foremost
goal is to gain or retain office- will, in an electoral context, be
motivated to support changes in war policy if opinion trends among their
constituents offer an incentive to do so'' (Lieberfeld, 2008). Public
opinion acts as a political constraint to the use of force, in addition
to providing probable policy platforms during upcoming elections.

For policymakers, understanding public opinion is crucial to making the
correct decision about when it is appropriate to use military force. In
addition, for citizens, public opinion can provide a good predictor of
future policy decisions. Therefore, this paper will attempt to create a
Frequentist and a Bayesian model to estimate the public opinion on
military intervention based on a series of surveys known as the
\emph{Chicago Council Survey of American Public Opinion on U.S.Foreign
Policy} from 2016 and 2017. The survey question that we will analyze
focuses on the public opinion on the effectiveness of military
intervention in achieving the foreign policy goals of the United States.
Using this data, we will estimate the expected political support in each
U.S. state in 2016 and 2017. This will be done using a method known as
multilevel regression with post-stratification, using data from the
American Community Survey. Then, this paper will attempt to validate
these models by checking that there are no errors in this model, and
observing if national surveys from 2015 to 2019 produce similar national
results to the results of this paper.

\hypertarget{background-and-significance}{%
\subsubsection{Background and
Significance}\label{background-and-significance}}

There have been numerous opinion polls and regressions that have been
run in order to estimate pro- or anti-war opinions during the
``post-cold war'' era. There are a few reasons why authors and academics
tend to start their analysis of war trends in the ``post-cold war'' era.
In general, the individual's personal policy preference on war tends to
be very temporal. The cold war provided a very unique circumstance, such
as the policy of deterrence and the fear of a global war, that does not
necessarily reflect the current circumstances of conflicts in the
``post-cold war'' era. Because of this, our scholarship tends to focus
on the ``post-cold war'' era as a separate era to the cold war era in
understanding the how the public perceives military intervention.

According to Gelpi and Feaver, the determination of a citizen's
preference is based on that individual's demographics, and the context
of the situation. Specifically, individuals tend to be in support of
military intervention if they believe that they have the ``Right'' to
attack, and they believe that military intervention will be successful
to achieve their preferred policy goals. Many of these contextual
results are likely to reflect incredibly temporal events that are
vulnerable to political changes. For instance, individuals tend to look
towards individuals such as politicians that they trust, and
organizations such as NATO. If an organization like NATO, or a president
that they trust endorse a military action, individuals are extremely
likely to support that military interaction. This was found by both
Khazan (2013) and by Gelpi and Feaver (2009).

In addition, individuals are likely to make choices depending on the
Principal Policy Objective, according to Jentleson (1992). This thesis
argues that individuals can evaluate the reasons why the U.S is
intervening, and that the primary objective of intervention is likely to
influence the level of political support. For instance, according to a
poll from Jentleson, individuals tend to be more in support of Foreign
Policy Restraint missions, as opposed to Internal Political Change
missions. Foreign Policy Restraint typically involves the use of
military force in order to affect the behavior of a nation, whereas
Internal Political Change Missions involve using the military to replace
or restructure an existing regime.

Finally, individuals also tend to support missions that they believe
will be successful. As noted by Gelpi and Feaver (2009), public support
for military intervention tends to drastically decrease during periods
of failure. For instance, in the aftermath of a failed raid in Somalia
in 1993, public support for military intervention dropped by 8
percentage points. In general, individuals tend to support a military
intervention when there is evidence that the intervention is likely to
be successful.

While these circumstantial predictors are important, there are
additional variables that might be important in order to predict pro-
versus anti-war sentiment. These variables are demographic, and tend to
be less dependent on the context of the conflict in determining public
opinion. For instance, our evidence of gender suggests that there is a
mixed effect of whether women or men are more supportive of conflict.
Gelpi and Feaver wrote \emph{Paying the Human Costs of War}, which
collects regressions that indicate that there is conflicting evidence
about the effect of gender in being pro- or anti-war. Men were more
supportive of military intervention in Lebanon in 1983. Men also tended
to be more supportive of military intervention in Somalia in 1993. In
the early 2000s, a general poll reported by Gelpi and Feaver stated that
there was no statistical significance in gender in determining whether
an individual supports military intervention. However, evidence from
polls taken during the Kosovo crisis suggests that Women were more
likely to support military intervention through air strikes in the
Kosovo War. Women were also more likely to fall into the category that
Gelpi and Feaver refer to as ``Timid Hawks''. Gelpi and Feaver use this
term to classify individuals who are supportive of war in the abstract,
but tend to withdraw their support as the conflict becomes costly, or as
casualties from the conflict increase. This is also important to note,
as it highlights a major theme of Gelpi and Feaver, which is that
casualty tolerance is not the same thing as conflict tolerance.

Another variable that is very important to consider is age. Anecdotally,
there is evidence that suggests that age can both increase and decrease
support for war. Many historians note the fervor that younger men tended
to have for World War 1. Meanwhile, Vietnam War Era protests are
remembered for the swarms of college students who protested against the
U.S's involvement in Southeast Asia. However, data tends to provide a
conflicted history of who tends to be more pro-war. Older individuals
were less likely to support military intervention in Lebanon and
Somalia. Older individuals were less likely to support the Iraq War
(Smith and Lindsay, 2003). However, age did not appear to play any
significant role in whether an individual supported intervention in the
Kosovo War. In general, the role that age plays in determining pro- and
anti-war support appears to be suggesting that, younger individuals are
more likely to support conflict. However, this is only if we consider
this to be a linear relationship.

The role of education in determining pro- and anti-war stance tends to
be very consistent. These estimates tend to suggest that individuals
with higher levels of education may be less likely to support war.
People were less likely to support a military intervention in Kosovo
with higher levels of education. In addition, people with higher levels
of education were less likely to support the use of force if the
objective of the mission is a matter of national security. However,
education did not have a significant change for whether an individual
supported military intervention in Yemen or Lebanon. In addition,
according to Smith and Lindsay (2003), individuals with a postgraduate
education were among the few groups to show majority opposition to the
Iraq War in late 2002. All of this evidence tends to suggest that
individuals with a high level of education are less likely to support
war.

There are many different additional variables that might be important in
determining which individuals will support or not support a war. For
instance, people tend to support a war if that war was initiated by the
party that aligns with their political identity. For instance, democrats
tend to be more supportive of military intervention if a Democratic
president initiated that war, and Republican voters are more likely to
support wars initiated during a Republican presidency. In addition,
individuals who are veterans tend to be less likely to support military
intervention, as found by Khazan (2013). The effect of other demographic
variables, such as race, tend to have inconsistent evidence on the
direction and the significance of the effect of racial identity on pro-
and anti-war stance.

\hypertarget{data}{%
\subsubsection{Data}\label{data}}

The first dataset that I used was the Chicago Council Survey of American
Public Opinion on U.S. Foreign Policy. This is an annual survey that
asks questions related to the foreign policies in the United States and
the public opinion on these policies. For my data, I am using the 2016
and 2017 Chicago Council Survey of American Public Opinion. These two
years were used because they are both publicly available, and the
response question is worded the same in both surveys. The target
population of this survey was those living in the U.S. who were
non-institutionalized and who are aged over 18. The survey was
administered by the KnowledgePanel, which is a probability-based web
panel designed to be representative of the U.S. The administrators of
this survey included various checks, such as removing those who answered
the test too quick and those who failed the ``quality checks''. The
``quality checks'' consisted of questions such as ``Pick option 3'',
which indicates whether a respondent was attentive during the survey.
The Chicago Council Survey has 4767 observations over the 2 years.

We took multiple steps to clean this data. First, any cases where there
was a missing value were discarded to prevent any bias that might occur
by keeping these variables in the dataset. In addition, many of the
variables were manipulated to create new variables for the purpose of a
regression. Initially, there was a variable called \emph{Q8\_14}. This
variable recorded the respondent's answer to the question \emph{How
effective do you think each of the following approaches are to achieving
the foreign policy goals of the United States- very effective, somewhat
effective, not very effective, or not effective at all: Intervening
Militarily}. Using this variable, we manipulated it into a new variable
called \emph{ProWarBinary}. If this variable is equal to 1, then the
respondent believes that military intervention is either very effective
or somewhat effective. Meanwhile, if the respondent believes that
military intervention is not very effective or not effective at all,
then the variable will return with a value of 0.

In addition, the Chicago Foreign Council data also had additional
variables that reflect the demographics of the respondents. In this
case, the variables includes information about the racial, political,
social, economic and family history of the respondent. There are a few
variables of note that will later be used in the model to build the
multilevel regression with post-stratification. The variable
\emph{EducationBracket3} represents a subset of different levels of
education. This includes a variables for those who did not get a high
school diploma, those with a high school diploma, those with some
college, those with a bachelor's degree, and those with more education
than a bachelor's degree.

Another variable of note is \emph{AgeBracket}. This variables indicates
what age range the respondent is in. Respondents can either be in a
stratum for those aged 18 to 24, 25 to 34, 35 to 44, 45 to 64, and for
those who are 65 or older. These strata were constructed to match the
format of the strata that are reported in the ACS. In addition, the
Chicago Foreign Council also included information about what state a
respondent was from, reported in the variable titled \emph{State}.

There are some checks that we may want to do to evaluate whether there
was any response bias. In presenting the question represented in
variable \emph{Q8\_14}. In the survey, this response could have been
shown anywhere between the first and the eighth statement. We might
suspect that the order that military intervention is presented will
influence the response and create response bias. One method can be
checking if an individual thinks military intervention is effective or
not depending on when the respondent is presented with the policy choice
of military intervention.

\begin{figure}
\centering
\includegraphics{Report_files/figure-latex/unnamed-chunk-4-1.pdf}
\caption{Proportion of Belief in Effectiveness of Intervention by
Statement Order}
\end{figure}

Figure 1 illustrates the response of the perception of the effectiveness
of military intervention, based on the Variable \emph{Pro War Factor}.
There is not a lot of variation in the responses of the effectiveness of
military intervention depending on the position that the statement was
shown in. For instance, we can see that people who were shown this
statement earlier may report being more in favor of intervention, as
opposed to those who were asked about military intervention later.
However, it appears like the position that the statement was shown in
does not carry a significant level of bias, and therefore we can
consider that source of response bias as insignificant.

We can also observe other relationships in our data. One particular
question might be whether individuals have different levels of support
based on what region they live in. Different cultural values and
different regional relationships with the military may determine the
different levels of support for military conflict. For instance, Karol
and Miguel (2007) tested whether different regions had different levels
of political support for reelecting George W. Bush based on the number
on how many casualties had occurred from citizens in each region, based
on the speculation that different geographic regions had different
relationships with the war. Therefore, support for intervention within
each U.S. region may be important to determining if there is a different
level of support depending on geographic region.

\begin{figure}
\centering
\includegraphics{Report_files/figure-latex/unnamed-chunk-5-1.pdf}
\caption{Belief in the Effectiveness of Intervention by Region of
Residence}
\end{figure}

In Figure 2, we can see that there does appear to be some slight
differences between different regions in determining the beliefs of
effectiveness of confclit The New England region has less support for
conflict than the other regions. However, the West-South Central region
and the East-South Central Region both tend to be more supportive of
conflict than other regions. However, the significance of these
responses is not clear based on Figure 2.

In addition, we are able to observe whether there are differences in
pro-conflict and anti-intervention sentiment based on education. This is
observed in Figure 3, which indicates that there is some correlation
between education level and belief in the effectiveness of military
intervention. For people who did not receive a high school diploma,
roughly 75\% of respondents believed that military intervention is an
effective method to achieve the U.S's foreign policy goals. However, for
those with a postgraduate degree, about 50\% of respondents believed
that military intervention is likely to achieve the U.S's foreign policy
goals. In addition, Figure 4 illustrates that there is some potential
relationship between age and support for military intervention. Those
who were between ages 18-24 and 45-64 are more likely to support
intervention, relative to other age brackets. However, this relationship
does not appear to be as strong as other relationships.

\begin{figure}
\centering
\includegraphics{Report_files/figure-latex/unnamed-chunk-6-1.pdf}
\caption{Belief in the Effectiveness of Intervention by Level of
Education}
\end{figure}

\begin{figure}
\centering
\includegraphics{Report_files/figure-latex/unnamed-chunk-7-1.pdf}
\caption{Belief in the Effectiveness of Intervention by Age}
\end{figure}

Another dataset that was used in this analysis is the American Community
Survey. The American Community Survey, or ACS, is a survey that has
information on individuals and households throughout the country. In
particular, this data focuses on the demographics, occupations,
education, and other background info of any respondents. The data that
was used in this paper was the summary statistic named \emph{B15001},
from the 5-Year ACS between 2015 and 2019. This data includes the raw
count of number of respondents who were in a certain strata defined by
gender, education level, and age bracket. These strata are pairwise
disjoint, individuals cannot belong to multiple strata within the
structure of the ACS. Then, I created a new variable, titled \emph{TVP},
standing for Total Voting Population. This was the sum of all of the
strata. I then divided each stratum by the variable \emph{TVP}, in order
to get an estimation of the proportion of each stratum within each
state.

The ACS does have a few problems with it that make it a suboptimal
resource to estimate proportions. The ACS does not ask questions about
values such as what political party an individual identifies with, or
what foreign policy goals the U.S. should have. These questions may be
important in relation to pro- and ant-conflict sentiment. These problems
will be of particular concern during the implementation of a multilevel
regression with post-stratification model.

\hypertarget{methods}{%
\subsubsection{Methods}\label{methods}}

This paper will focus on using a multilevel regression with
post-stratification, or MRP. A multilevel regression with
post-stratification is a method to estimate public opinion for areas
that are smaller than the national level, such as individual states. It
is highly unlikely that there will be multiple state surveys that all
ask the same questions using similar wording and surveying methods.
Because of this, statisticians have developed strategies to use national
level opinion polls to estimate public opinion for smaller geographic
areas. One method is disaggregation, which is where national surveys are
pooled and then sorted by region to calculate public opinion. However,
MRP has become an alternative method to disaggregation. MRP has been
shown to outperform disaggregation with small and medium sized samples
and can estimate state public opinion with a single large national poll
of roughly 1,400 respondents(Gelman, 2018). MRP is better suited for
estimating public opinion issues that are swayed heavily by
circumstances. In addition, MRP is useful for estimating the responses
in smaller states that do not have a large number of respondents who
reside in those states. Because of these advantages, building an MRP
model is preferable to using a model of disaggregation to estimate the
beliefs on the effectiveness of military intervention in each U.S.
State. Multilevel regression with post-stratification has been used
frequently by statisticians to estimate public opinion by geographic
area. For instance, Bohr (2014) used MRP to estimate the probability of
an individual in each U.S. county believing in climate change, and
understanding their preferences for environmental policy. In addition,
Wang (2014) used MRP in order to estimate the voters' preferences for
the U.S. election in 2012 by using surveys submitted through the online
platform Xbox Live, which estimated the share of votes for Barack Obama
with a margin of error of .6 percentage points.

There are a few steps needed in order to create an MRP. First, data
needs to be collected that focuses on public opinion data, and different
potential predictors. Next, there would need to be census data collected
in order to estimate the proportion of different strata in each state.
We would then estimate the probability of believing U.S. intervention is
effective within each stratum based on our public opinion data, using a
logistic regression from the Chicago Council Surveys. Then, using our
estimates of the probability of any person within a strata being pro
conflict, we post-stratify these models using the proportional size of
the strata relative to the rest of the state.

One important detail in an MRP is that the design of the logistic
regression and the design of the ACS must be cohesive in order to
effectively implement an MRP. An MRP works because it is accurately able
to assign each possible predicted response based on our regression to
our weights for post-stratification. Therefore, there would be
significant problems in using a multilevel regression with
post-stratification if the multilevel regression includes terms that are
not included in poststratify weights. This would result in us having
predicted response probabilities without any understanding of what the
approximate weights of these responses will be.

Therefore, in building our logistical regression, we will attempt to
reduce the AIC value and we will only build a model where we can also
estimate the weights of these different responses based on the ACS data.
In addition, the model that we analyze will also focus on the most
important aspects that may have been identified in our data section. For
instance, one important variable to use in our regressions would be
education. Education level is a variable that shows a clear relationship
to pro vs.~anti-war sentiment. There are roughly 1000 different distinct
counts that exist in the ACS data. However, only about 30 different
types of counts include educational attainment as a key variable. In
addition, very few of these variables account for the full voting aged
population.

The best model that I could find that includes variables that could be
weighted in the census was a model which used educational attainment,
gender, and age brackets. However, the state was also used as a random
effect. This model produced the smallest AIC of any adequate model that
was built in this experiment, at a value of 6130.953{[}1{]}. We can
therefore write the estimated model as a two-part section. The first
part represents our post-stratification, where we multiply the
probability of support of intervention by each stratum within the state
by the probability of any given person in that state being within that
stratum. We then sum the product of all these probabilities and weights
in order to estimate the probability of support within each state. The
second line through the fourth line represents our multilevel model,
which includes a random effect for the state, in order to account for
variations in public opinion within each state that are not accounted
for by our model. Finally, our estimation of the probability of being
within each strata is defined by calculating the percent of people
within each state \(j\), who were in stratum \(i\) as a percent of the
total voting population.

\begin{flalign}
& P(\theta_{j}=1|\text{State=j})=\sum_{i=1}^{50}P(\theta_{ij}=1|\text{Stratum}=i)*P(\text{Stratum}=i|\text{State}=j) &\\
& P(\theta_{ij}=1|\text{Strata}_{j}=i)=\text{logit}^{-1}(\alpha+\beta_{1}*\text{Male}_{i}+\beta_{2}*\text{High School Degree}_{i} &\\
& +\beta_{3}*\text{Some College}_{i}+\beta_{4}*\text{Bachelor Degree}_{i}+\beta_{5}*\text{Graduate Degree}_{i}+\beta_{6}*\text{25 to 34}_{i} &\\
& +\beta_{7}*\text{35 to 44}_{i}+\beta_{8}*\text{45 to 64}_{i}+\beta_9*\text{65+}_{i}+u_{j}+\epsilon{ij}) &\\
& u_{j}\sim \mathcal{N}(0,\,\sigma_{u}^{2}) &\\
& \epsilon{ij}\sim \mathcal{N}(0,\,\sigma_{\epsilon}^{2}) &\\
& P(\text{Strata}=i|\text{State}=j)=\frac{\text{Population in Stratum i in State j}}{\text{Total Voting Population in State j}} &
\end{flalign}

\(\theta\) represents whether an individual believes that military
intervention is effective or ineffective at achieving the U.S.'s policy
goals. \(\alpha\) represents the intercept, which would calculate the
probability a women who is between the ages of 18 to 24 with less than a
high school education believes that military intervention is effective.
Each of the subsequent \(\beta\)'s then represent the effect of
individual effects, such as identifying as a man, or having a Graduate
Education, on pro-intervention sentiment. The value \(u_j\) is the
random effect that being within a certain state has on their beliefs of
intervention, while \(\epsilon_{ij}\) is the random error term.

By construction, our stratum will be pairwise disjoint and they will
cover the voting aged population. The stratum are broken up into
categories based on age, sex, and educational obtainment. In addition,
it is not possible for an individual in the voting age population to not
fit into any of the stratum, or for them to fit into multiple strata,
based on the construction of the ACS and the Chicago Council survey.
This is important to ensuring that our MRP model will be successful in
estimating the probability of an individual's beliefs on military
intervention within a certain state.

This paper will also build a logistic regression under a Bayesian
framework. This would require us to set our priors. These priors would
be based on the background research from the regressions that are in
Gelpi and Feaver (2009). Using logistic regressions on public opinion on
conflict, we should be able to have an estimate of the effects of
gender, education, and age on an individual's beliefs on military
intervention.

The expected prior for the effect of beliefs on military intervention
for gender is rather complex. For instance, traditional studies that
focused on war, particularly in the 1980s and 1990s, tend to suggest
that men tend to be more supportive of conflicts. However, this is
contrasted with some contemporary data. For instance, women tended to be
more likely to be classified as ``Timid Hawks''. In addition, women
tended to have higher levels of support for military intervention is
Kosovo, and for humanitarian missions in Yemen. There are two very
important distinctions to be made in our previous data, the age of the
data that we are observing, and how similar these surveys are to the
Chicago Council survey question that we analyzed. For instance, more
recent data tends to suggest women may be more likely to believe in
conflict than men. In addition, when asked about military intervention
before it happened, women tended to be more likely to support
intervention, as seen in supporting intervention in Kosovo and Yemen.
This is similar to the structure of the Chicago Foreign Council
question, as it asks for the respondent's level of support before any
military intervention has occurred. Because of this bias, there is some
evidence that men may be slightly less likely to believe that war is
effective in the abstract. Because of this, it might make sense to
assign that the prior for gender is a vague prior with a slight
preference to show that women may prefer military conflict. Because of
this, a prior of \(Male\sim N(-.1,1)\) may be appropriate. The standard
deviation suggests that we believe it is plausible that the effect could
be far greater, or that the effect could suggest that men are more
likely to support conflict.

Another important prior to set is the expectations for the priors by age
bracket. Most of these surveys use age as a continuous variable, instead
of establishing different age brackets. However, these linear
relationships tend to suggest that the effect of age on support for
military intervention tends to be roughly close to 0, or suggest that
younger individuals are more likely to support military intervention. In
many of these regressions, the effect of age was also not statistically
significant. Therefore, there may be some value in establishing a vague
prior, since there is not strong evidence establishing a trend. In
addition, we are constructing a categorical variable, as opposed to
other regressions which have treated age as a linear effect. Therefore,
we might decide to use a prior, where for each age bracket we assume
that the coefficient has a normal prior, with a mean of zero and a
standard deviation of 1.

Finally, a prior would need to be set for the effect of education on
beliefs of military intervention. In older studies, such as examining
support for conflicts in Lebanon and Somalia, there was not a very
strong relationship between the level of education and an individual's
preference for military conflict. However, In Kosovo, higher levels of
education lead to an individual to be less likely to support military
intervention. In addition, individuals were more likely be classified as
``timid hawks'' if they had more education. Because of these
relationships, we might expect that the prior is that higher levels of
education result in a lower likelihood of supporting war. Arbitrarily,
some priors can therefore be set. For instance, we may establish that
every subsequent level of education will affect education by -.2.
Therefore, we might establish that each of our levels of education has
an independent prior, as established below.

\begin{flalign}
&\text{High School Graduate}\sim N(-.2,1) &\\
& \text{Some College} \sim N(-.4,1) &\\
& \text{Bacehlor Degree} \sim N(-.6,1) &\\
& \text{Graduate Degree} \sim N(-.8,1) &
\end{flalign}

\hypertarget{frequentist-results}{%
\subsubsection{Frequentist Results}\label{frequentist-results}}

On Table 1, we are able to see the effects of each standard effect in
our model, as outlined previously on our equation through lines 2 to 4.
We are able to see that the estimated intercept on this equation is
1.1191. This indicates that we expect that on average, a women aged
between 18-24 without a high school diploma is 75.38\% likely to believe
that military intervention is effective, before accounting for the
random effect of the state. According to our data from the Chicago
Council, all else held equal, men are 23.5\% less likely to believe that
military intervention is an effective method of achieving the U.S.'s
policy goals. These results are statistically significant.

The effects of education are also evident, based on our model. A
respondent who had graduated from high school was 21.56\% less likely to
believe that military intervention is effective in achieving the U.S.'s
goals, when compared to individuals who did not achieve a high school
education. However, these results were not statistically significant at
the 5\% level. Meanwhile, respondents who attended some college were
38.07\% less likely to believe military intervention is effective when
compared to those who did not achieve a high school education.
Respondents who got a bachelor's degree were 46.31\% less likely to
believe that military intervention was effective compared to those who
did not get a high school diploma. Finally, respondents who had a
postgraduate education were 64.67\% less likely to believe that military
intervention is effective compared to those who did not get a high
school diploma. All of these results were statistically significant at
the 5\% level.

\begin{longtable}[]{@{}
  >{\raggedright\arraybackslash}p{(\columnwidth - 8\tabcolsep) * \real{0.3088}}
  >{\raggedleft\arraybackslash}p{(\columnwidth - 8\tabcolsep) * \real{0.1324}}
  >{\raggedleft\arraybackslash}p{(\columnwidth - 8\tabcolsep) * \real{0.1618}}
  >{\raggedleft\arraybackslash}p{(\columnwidth - 8\tabcolsep) * \real{0.1176}}
  >{\raggedleft\arraybackslash}p{(\columnwidth - 8\tabcolsep) * \real{0.2794}}@{}}
\caption{Estimates of Fixed Effects}\tabularnewline
\toprule
\begin{minipage}[b]{\linewidth}\raggedright
\end{minipage} & \begin{minipage}[b]{\linewidth}\raggedleft
Estimate
\end{minipage} & \begin{minipage}[b]{\linewidth}\raggedleft
Std. Error
\end{minipage} & \begin{minipage}[b]{\linewidth}\raggedleft
z value
\end{minipage} & \begin{minipage}[b]{\linewidth}\raggedleft
Pr(\textgreater\textbar z\textbar)
\end{minipage} \\
\midrule
\endfirsthead
\toprule
\begin{minipage}[b]{\linewidth}\raggedright
\end{minipage} & \begin{minipage}[b]{\linewidth}\raggedleft
Estimate
\end{minipage} & \begin{minipage}[b]{\linewidth}\raggedleft
Std. Error
\end{minipage} & \begin{minipage}[b]{\linewidth}\raggedleft
z value
\end{minipage} & \begin{minipage}[b]{\linewidth}\raggedleft
Pr(\textgreater\textbar z\textbar)
\end{minipage} \\
\midrule
\endhead
Intercept & 1.1191 & 0.1580 & 7.0807 & 0.0000 \\
Male & -0.2683 & 0.0614 & -4.3684 & 0.0000 \\
High School Graduate & -0.2429 & 0.1351 & -1.7977 & 0.0722 \\
Some College & -0.4791 & 0.1338 & -3.5803 & 0.0003 \\
Bachelor's Degree & -0.6219 & 0.1390 & -4.4746 & 0.0000 \\
Graduate Degree & -1.0406 & 0.1432 & -7.2676 & 0.0000 \\
Age 25 to 34 & -0.0221 & 0.1342 & -0.1650 & 0.8689 \\
Age 35 to 44 & 0.1383 & 0.1405 & 0.9848 & 0.3247 \\
Age 45 to 64 & 0.2317 & 0.1236 & 1.8750 & 0.0608 \\
Age 65 or older & 0.0150 & 0.1281 & 0.1173 & 0.9067 \\
\bottomrule
\end{longtable}

The final set of fixed effects is related to the age of the respondent.
People who are in the age range from 25 to 34 are 2.19\% less likely to
believe that military conflict is an effective strategy to achieve the
U.S's foreign policy goals. However, these results are not statistically
significant. Respondents who are aged 35 to 44 were 14.83\% more likely
to believe that military intervention is effective, but this result was
not statistically significant. People aged 45 to 64 were 26.07\% more
likely to believe that military intervention, compared to people who
were aged 18 to 24. These results were only statistically significant at
the 10\% level.

Using our estimates, we can estimate what the potential public opinion
within each state may be. Therefore, this data is displayed on Figure 5.
Some states and regions tend to have a lower opinion of military
intervention than other states, as illustrated by the map. However, in
general, most states tend to have a level estimated 60 to 70\% of voting
aged adults within that state who believe that intervention is effective
at achieving the U.S's goals. States such as California, Massachusetts,
and Minnesota all have roughly 62\% of their citizens who believe that
military intervention is an effective strategy to achieve the U.S's
goals. Meanwhile, in Texas, roughly 68\% of citizens believe that
military intervention is an effective strategy to achieve the U.S's
foreign policy goals. Other states, such as Arkansas, North Carolina,
and Indiana, also have comparatively high levels of public belief that
military intervention is effective.

\begin{figure}
\centering
\includegraphics{Report_files/figure-latex/unnamed-chunk-9-1.pdf}
\caption{U.S. Map of Public Opinion on the Effectiveness of Military
Intervention under a Frequentist Model}
\end{figure}

\begin{figure}
\centering
\includegraphics{Report_files/figure-latex/unnamed-chunk-10-1.pdf}
\caption{Differences in estimates of Public Opinion within Each State,
from an MRP and a Simple Disaggregation Model}
\end{figure}

Another interesting detail will be observing the comparison of using MRP
and using disaggregation in order to estimate the probability that an
individual is supportive of military intervention. Disaggregation is a
blunt method, where we first break a national survey into each
subsequent, smaller region, and then use the raw percent of responses
from that subset of the survey to estimate public opinion without any
model. The difference between the MRP model and a simple disaggregation
model is illustrated in Figure 6. On this graph, points above the red
line indicate that the MRP estimate is greater than the estimate
provided by observing the percent who believed military intervention was
effective in the poll within each state. Likewise, points below the red
line indicate that the MRP-based estimate is lower than the estimate
provided by observing the percent who believed military intervention was
effective within each state. Within some states, such as Hawaii,
Vermont, Kansas, and North Dakota, the estimates are substantially
different between these two methods.

\hypertarget{bayesian-results}{%
\subsubsection{Bayesian Results}\label{bayesian-results}}

Table 2 illustrates the results of running a Bayesian model, using the
priors from the methods section. Before accounting for the random effect
of what state the respondent lives in, an 18 to 24 year old woman who
does not have a high school diploma has a 75.46\% probability of
believing that military intervention is an effective strategy to achieve
foreign goals. Men are 23.44\% less likely to support military
intervention, all else held equal. Meanwhile, high school graduates are
21.39\% less likely to believe that military intervention is effective.
Respondents who attended some college, obtained a bachelor's degree, and
attended beyond a bachelor's degree were 37.97\%, 46.17\%, and 64.52\%
less likely to believe that military intervention was effective
respectively, all else held equal.

The effects of age are also illustrated on Table 2. Respondents who are
aged 25 to 34 were 2.76\% less likely to believe that intervention was
effective when compared to those who were aged 18 to 24. Respondents
aged 35 to 44 were on average 14.02\% more likely to believe that
intervention was effective all else held equal. Respondents who were
aged 45 to 64 were 25.23\% more likely to believe that intervention was
effective, while respondents who were aged 65 or older were .93\% more
likely to believe that intervention is effective.

The results from the Bayesian analysis tend to be rather similar to the
results from our frequentist model. This may be because the priors that
were used were vague. In a logistic regression a standard deviation of
1, as used in our prior, then the majority of the information on this
regression will be similar to our data model if we used a simple
frequentist model.

\begin{longtable}[]{@{}
  >{\raggedright\arraybackslash}p{(\columnwidth - 14\tabcolsep) * \real{0.2530}}
  >{\raggedleft\arraybackslash}p{(\columnwidth - 14\tabcolsep) * \real{0.0964}}
  >{\raggedleft\arraybackslash}p{(\columnwidth - 14\tabcolsep) * \real{0.0843}}
  >{\raggedleft\arraybackslash}p{(\columnwidth - 14\tabcolsep) * \real{0.1566}}
  >{\raggedleft\arraybackslash}p{(\columnwidth - 14\tabcolsep) * \real{0.0964}}
  >{\raggedleft\arraybackslash}p{(\columnwidth - 14\tabcolsep) * \real{0.1566}}
  >{\raggedleft\arraybackslash}p{(\columnwidth - 14\tabcolsep) * \real{0.0723}}
  >{\raggedleft\arraybackslash}p{(\columnwidth - 14\tabcolsep) * \real{0.0843}}@{}}
\caption{Estimates of Fixed Effects}\tabularnewline
\toprule
\begin{minipage}[b]{\linewidth}\raggedright
\end{minipage} & \begin{minipage}[b]{\linewidth}\raggedleft
Mean
\end{minipage} & \begin{minipage}[b]{\linewidth}\raggedleft
SD
\end{minipage} & \begin{minipage}[b]{\linewidth}\raggedleft
10\% Quantile
\end{minipage} & \begin{minipage}[b]{\linewidth}\raggedleft
Median
\end{minipage} & \begin{minipage}[b]{\linewidth}\raggedleft
90\% Quantile
\end{minipage} & \begin{minipage}[b]{\linewidth}\raggedleft
n\_eff
\end{minipage} & \begin{minipage}[b]{\linewidth}\raggedleft
Rhat
\end{minipage} \\
\midrule
\endfirsthead
\toprule
\begin{minipage}[b]{\linewidth}\raggedright
\end{minipage} & \begin{minipage}[b]{\linewidth}\raggedleft
Mean
\end{minipage} & \begin{minipage}[b]{\linewidth}\raggedleft
SD
\end{minipage} & \begin{minipage}[b]{\linewidth}\raggedleft
10\% Quantile
\end{minipage} & \begin{minipage}[b]{\linewidth}\raggedleft
Median
\end{minipage} & \begin{minipage}[b]{\linewidth}\raggedleft
90\% Quantile
\end{minipage} & \begin{minipage}[b]{\linewidth}\raggedleft
n\_eff
\end{minipage} & \begin{minipage}[b]{\linewidth}\raggedleft
Rhat
\end{minipage} \\
\midrule
\endhead
Intercept & 1.1235 & 0.1569 & 0.9208 & 1.1226 & 1.3240 & 9115 &
1.0002 \\
Male & -0.2672 & 0.0622 & -0.3472 & -0.2671 & -0.1879 & 20092 &
0.9997 \\
High School Graduate & -0.2407 & 0.1349 & -0.4151 & -0.2401 & -0.0660 &
8429 & 0.9997 \\
Some College & -0.4776 & 0.1347 & -0.6506 & -0.4777 & -0.3052 & 8376 &
0.9998 \\
Bachelor's Degree & -0.6194 & 0.1394 & -0.7981 & -0.6181 & -0.4391 &
8093 & 0.9997 \\
Graduate Degree & -1.0363 & 0.1435 & -1.2209 & -1.0366 & -0.8548 & 8024
& 0.9998 \\
Age 25 to 34 & -0.0280 & 0.1313 & -0.1948 & -0.0279 & 0.1412 & 8223 &
1.0006 \\
Age 35 to 44 & 0.1312 & 0.1371 & -0.0435 & 0.1301 & 0.3069 & 8433 &
1.0004 \\
Age 45 to 64 & 0.2250 & 0.1213 & 0.0707 & 0.2249 & 0.3796 & 7666 &
1.0006 \\
Age 65 or older & 0.0093 & 0.1262 & -0.1516 & 0.0078 & 0.1709 & 8151 &
1.0003 \\
\bottomrule
\end{longtable}

Using these results from the Bayesian analysis, we can then undergo the
same MRP process to estimate the public opinion within each state by
estimating the expected value within each stratum. The results of this
analysis are shown in Figure 7. The results of these estimates are
similar to the estimates for each state generated under the frequentist
model. States such as Texas, North Carolina, Arkansas, and Mississippi
all have larger percentages of their voting aged populations who believe
that military intervention is effective. In each of these states, we
estimate that between 67\% to 68\% of their population believe that war
is effective. However, states such as Massachusetts, California, and
Minnesota all have estimates that between 62\% to 63\% of their
population believe that military intervention is effective.

\begin{figure}
\centering
\includegraphics{Report_files/figure-latex/unnamed-chunk-12-1.pdf}
\caption{U.S. Map of Public Opinion on the Effectiveness of Military
Intervention under a Bayesian Model}
\end{figure}

To ensure that our model has blended well, we need to observe the R-Hat
Convergence values. This value, also known as the Gelman-Rubin
Convergence Diagnostic, evaluates the level of between chain convergence
and within chain convergence of the estimates for each predictor. If
this value is above 1.05, then it is expected that the models are not
converging well, which suggests that there are significant problems. On
Figure 8, the R-Hat score for each predictor is shown. Here, we can see
that the R-Hat suggests that every predictor has converged well in this
Bayesian model.

\begin{figure}
\centering
\includegraphics{Report_files/figure-latex/unnamed-chunk-13-1.pdf}
\caption{R-Hat Measures for each Effect to Estimate Between- and
Within-Chain Covergence}
\end{figure}

Another important detail to analyze is the credible intervals of the
Bayesian predictors. The credible intervals are displayed by observing
Figure 9. This figure displays the credible intervals. By observing
these credible intervals, we can note whether we expect the effects of
each predictor to have a statistically significant effect. For instance,
we can see that gender and education-related variables tend to be
statistically significant at the 5\% level. The exception of the
statistical significance of education-related effects is the difference
between those who did not complete high school, and those who completed
high school. However, age-based predictors are not statistically
significant, based on Figure 9.

\begin{figure}
\centering
\includegraphics{Report_files/figure-latex/unnamed-chunk-14-1.pdf}
\caption{The 95\% Credible Intervals of Estimates for the Coefficients
in the Bayesian Model}
\end{figure}

\hypertarget{discussion-and-conclusions}{%
\subsubsection{Discussion and
Conclusions}\label{discussion-and-conclusions}}

In both our frequentist and Bayesian approaches to estimate state-level
public opinion, we found that most states have between 60\% and 70\%.
State level opinion polls on military intervention beliefs are scarce,
as most polls do not focus on state-level responses. In addition, most
polls do not have questions that align well with the Chicago Council
Question ``\emph{How effective do you think each of the following
approaches are to achieving the foreign policy goals of the United
States- very effective, somewhat effective, not very effective, or not
effective at all: Intervening Militarily}''. Because of these problems,
it may be hard to compare individual estimates of each state in our MRP
model with other estimates. However, we could observe contemporary polls
to observe what percent of people are in support of military
intervention. In 2016 and 2017, during the period of polling, the
primary concern of military intervention was the focus on military
intervention in Syria against the Islamic State.

Based on the estimates of our Bayesian and frequentist model, weighted
by the total voting population of each state, the estimated percent of
national voting aged adults who believe that military intervention is
effective is expected to be roughly 65\%. We can then compare these
results to contemporary, national surveys from a similar time period.
For instance, according to Pew Research, 63\% of adults supported
Military Action against Islamic Militants in Iraq and Syria in 2015. In
2017, there is more evidence that individuals were in support of
military intervention in Syria. According to the Morning Consult, in
2017, nearly 57\% of Americans supported airstrikes and cyberattacks
against Syria, while 58\% supported establishing a no-fly zone. The
airstrikes in retaliation to a chemical attack by Bashar Assad against
Syrians in April 2017 were supported by 66\% of voters. 58\% of voters
supported the airstrikes in 2018, while 19\% of those who were polled
had no opinion on the airstrikes. Finally, in 2019, surveys suggested
that 60\% of Voters were in favor of keeping U.S. troops in Syria. Data
taken on public opinion for military intervention in Syria is rather
similar to the national estimates estimated by our MRP methods.

Beyond the validity of this study, there are additional weaknesses that
must be noted, including the difficulties of this method, and the
shortcomings of this paper. Some of the problems that we encounter
include the requirement for our data on the survey, and the data on the
population of each stratum must be constructed similarly. Because of
this, we are limited in the variables that we might choose for the model
that we may choose to build. For instance, we were not able to analyze
the impact of political orientation because these are not variables that
are collected by the census. In addition, some variables are not
packaged together. For instance, there was no variable that created
strata in the ACS from different variables such as racial identity,
veterans status, education level, and age. Therefore, in building our
model, we were restricted by the ACS data. One potential solution, and
opportunity for future research, is to use another form of Census, such
as the Public Use Microdata, to create strata outside of the prebuilt
strata included in the ACS. Because of time and computing restraints, I
was not able to use the Public Use Microdata to create adequate strata,
however, future researchers may want to adjust this research using this
resource.

Another set of limitations from this data is the limitations presented
by the question of interest. The question ``\emph{How effective do you
think each of the following approaches are to achieving the foreign
policy goals of the United States- very effective, somewhat effective,
not very effective, or not effective at all: Intervening Militarily}''
has three significant problems caused by vagueness. The first source of
vagueness is the term ``Intervening Militarily''. As noted in both Haas
(1994) and Gelpi et al (2009), there are many different methods of
Intervening Militarily. For instance, militaries may have troops
stationed overseas, such as in military stations. However, military
intervention may also look like using U.S. troops to train foreign
soldiers, establishing no-fly zones, using drones for airstrikes, or a
ground invasion. All of these different military actions are likely to
inspire different levels of public support.

In addition, another source of vagueness is ``the foreign policy goals
of the U.S.''. In the Chicago Survey, the surveyors list 16 different
possible foreign policy goals the U.S. might have, including Combating
International Terrorism, Protecting U.S. Jobs, Combating World Hunger,
and Defending our Allies' Security. It is very reasonable that, whatever
foreign policy goal is specified may change the response of public
opinion on the effectiveness of military intervention to achieve these
goals. For instance, respondents may believe that military intervention
is appropriate to Combat International Terrorism, but not to Defend our
Allies' Security. Without the specific foreign policy goal specified,
the question must be interpreted as whether military intervention is in
general a good method to achieve these goals.

Finally, one last source of vagueness is the fact that this question
does not specifically indicate the circumstances of military
intervention. It is entirely conceivable, as noted by Gelpi et
al.~(2009) and Jentleson (1992) that individuals will give or revoke
public support based on circumstantial variables. For instance,
individuals might be supportive of military intervention based on the
perceived threat of non-intervention is, and the likelihood of military
success.

Because of the vagueness of the question that we are estimating, the
interpretations that we are able to make off the results of this paper
are limited. We cannot conclude information about public support in
specific the case of specific military intervention methods, specific
policy goals, or specific circumstances of the conflict. Therefore, the
results of this paper can best be seen as an abstract concept of general
support and belief in military intervention as an effective policy
choice. This could be seen as a reflection of public belief in one of
St.~Augustine's ``Just War'' axioms, where we are observing the public's
belief that conflict is likely to achieve a worthy cause.

In this paper, we used data from the Chicago Council Survey of American
Public Opinion on U.S. Foreign Policy and the American Community Survey
in order to estimate public support for U.S. military intervention among
voting aged U.S. residents. The method used is known as a multilevel
regression with post-stratification, which uses larger national polls to
estimate the percent of support within each U.S. state. In general, we
found in most states that 60\% to 70\% of voting aged residents believed
that military intervention is effective at achieving the U.S's foreign
policy goals. These results were then further supported by contemporary
polls that suggest that U.S. support for intervention tended to be
around 60\% from 2015 to 2019.

\newpage

\hypertarget{references}{%
\subsubsection{References}\label{references}}

Alexander, R. (2022). Telling Stories with Data.
\url{https://tellingstorieswithdata.com/}

American National Election Studies. 2021. ANES 2020 Time Series Study
Preliminary Release: Combined Pre-Election and Post-Election Data
{[}dataset and documentation{]}. March 24, 2021 version.
www.electionstudies.org

British Broadcasting Corporation. (2014, November 11). The teenage
soldiers of World War One. BBC News. Retrieved May 20, 2022, from
\url{https://www.bbc.com/news/magazine-29934965}

Brown University. (2021, September 1). Costs of the 20-Year War on
terror: \$8 trillion and 900,000 deaths. Brown University. Retrieved May
17, 2022, from \url{https://www.brown.edu/news/2021-09-01/costsofwar}

Bohr, J. (2014). Public Views on the Dangers and Importance of Climate
Change: Predicting Climate Change Beliefs in the United States through
income moderated by party identification. Climate Change, 126, 217--227.

Gallup, G. (1942). How Important is Public Opinion in Time of War.
Proceedings of the American Philosophical Society, 85(5), 440--444.

Gelman, A., \& Little, T. (1997). Poststratication Into Many Categories
Using Hierarchical Logistic Regression. Survey Methodology.
\url{http://www.stat.columbia.edu/~gelman/research/published/poststrat3.pdf}

Gelman, A., Carlin, J. B., Stern, H. S., Dunson, D. B., Vehtari, A., and
Rubin, D. B. (2013). Bayesian Data Analysis. Chapman \& Hall/CRC Press,
London, third edition. (Ch. 6)

Gelman, A., Lax, J., Phillips, J., Gabry, J., \& Trangucci, R. (2018).
Using Multilevel Regression and Poststratification to Estimate Dynamic
Public Opinion. Columbia.
\url{http://www.stat.columbia.edu/~gelman/research/unpublished/MRT(1).pdf}

Gelpi, C., Feaver, P., \& Reifler, J. (2009). Paying the Human Costs of
War. Princeton University Press.

Haas, R. (1994). Intervention: The Use of American Military Force in the
Post-Cold War World. Carnegie Endowment for International Peace.

Jentleson, B. W. (1992). The Pretty Prudent Public: Post Post-Vietnam
American Opinion on the Use of Military Force. International Studies
Quarterly, 36(1), 49--73. \url{https://doi.org/10.2307/2600916}

Johnson, A., Ott, M., \& Dogucu, M. (2021). Bayes Rules! An Introduction
to Applied Bayesian Modeling. Chapman and Hall.

Karol, D., \& Miguel, E. (2007). The Electoral Cost of War: Iraq
Casualties and the 2004 Presidential Election. The Journal of Politics,
69(3), 633--648.

Kastellec, J., Lax, J., \& Phillips, J. (2019). Estimating State Public
Opinion With Multi-Level Regression and Poststratification using R.
Princeton.

Katkov, M., Taylor, J., \& Bowman, T. (2017, April 6). Trump orders
Syria airstrikes after `Assad choked out the lives' of civilians. NPR.
Retrieved May 30, 2022, from
\url{https://www.npr.org/2017/04/06/522948481/u-s-launches-airstrikes-against-syria-after-chemical-attack}

Khazan, O. (2013, September 4). What are the big factors determining
whether Americans support war? The Atlantic. Retrieved March 29, 2022,
from
\url{https://www.theatlantic.com/politics/archive/2013/09/what-are-the-big-factors-determining-whether-americans-support-war/279290/-sets-box-office-record}

Lieberfeld, D. (2008). What Makes An Effective Antiwar Movement?
Theme-Issue Introduction. International Journal of Peace Studies, 13(1),
1--14. \url{http://www.jstor.org/stable/41852966}

Pew Research Center. (2015, July 22). A year later, U.S. campaign
against Isis Garners Support, Raises Concerns. Pew Research Center -
U.S. Politics \& Policy. Retrieved May 30, 2022, from
\url{https://www.pewresearch.org/politics/2015/07/22/a-year-later-u-s-campaign-against-isis-garners-support-raises-concerns/}

Sheffield, M. (2019, January 17). Poll: Most Americans want US troops in
Syria. The Hill. Retrieved May 30, 2022, from
\url{https://thehill.com/hilltv/what-americas-thinking/425803-poll-most-americans-still-want-a-us-military-presence-in-syria/}

Shepard, S. (2018, April 18). Majority supports Syrian airstrikes.
Politico. Retrieved May 30, 2022, from
\url{https://www.politico.com/story/2018/04/18/syria-airstrikes-trump-poll-530293}

Smeltz, Dina, Friedhoff, Karl, Kafura, Craig J., Holyk, Gregory, and
Busby, Joshua W. 2016 Chicago Council Survey of American Public Opinion
on U.S. Foreign Policy. Ann Arbor, MI: Inter-university Consortium for
Political and Social Research {[}distributor{]}, 2018-04-13.
\url{https://doi.org/10.3886/ICPSR36806.v1}

Smeltz, Dina, Daalder, Ivo, Friedhoff, Karl, and Kafura, Craig. 2017
Chicago Council Survey of American Public Opinion on U.S. Foreign
Policy. Inter-university Consortium for Political and Social Research
{[}distributor{]}, 2021-06-21.
\url{https://doi.org/10.3886/ICPSR37970.v1}

Smith, C., \& Lindsay, J. M. (2003, June 1). Rally 'Round the flag:
Opinion in the United States before and after the Iraq War. Brookings.
Retrieved March 29, 2022, from
\url{https://www.brookings.edu/articles/rally-round-the-flag-opinion-in-the-united-states-before-and-after-the-iraq-war/}

Wang,W., et al., Forecasting elections with non-representative polls.
International Journal of Forecasting (2014),
\url{http://dx.doi.org/1-.1016/j.ijforecast.2014.06.001}

Yokley, E. (2017, April 12). Americans Trust Trump the Most to End
Syrian Conflict. Morning Consult. Retrieved May 30, 2022, from
\url{https://morningconsult.com/2017/04/12/americans-trust-trump-end-syrian-conflict/}

\newpage

\hypertarget{appendix}{%
\subsection{Appendix}\label{appendix}}

{[}1{]} There were other models that did produce AIC. However, these
would not have allowed us to get a comprehensive set of weights from the
ACS. For instance, some of these models would have produced estimations
that we would not have any adequate weights to use in order to
poststratify, according to the ACS summary statistics. Another case
could be seen by the Model titled ``M25b''. This model included the
effect of ethnicity. However, it would have not allowed us to provide
weights for any individual who was between the ages of 18 and 25. This
would be a large chunk of the voting population that the model would not
have adequate weights for, which would be a problem.

For information on the coding of this project, as well as the data files
that were used in this project, please go to

\end{document}
